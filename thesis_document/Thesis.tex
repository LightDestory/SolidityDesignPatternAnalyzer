\documentclass[12pt,oneside,a4paper,PisaPhdThesis, italian]{PhdThesis}
\usepackage[centertags]{amsmath}
\usepackage{framed}
\usepackage{libertine}
\usepackage{csquotes}
\usepackage[english, italian]{babel}
\usepackage[font=small,labelfont=bf,tableposition=top]{caption}
\usepackage{graphics}
\usepackage{graphicx}
\usepackage{color}
\usepackage{hyperref}
\usepackage{epsfig}
\usepackage{amsfonts}
\usepackage{amssymb}
\usepackage{amsthm}
\usepackage{rawfonts}
\usepackage{enumerate}
\usepackage{capt-of}
\usepackage{url}
\usepackage{xspace}
\usepackage{xthesis} 
\usepackage{xtocinc} 

%Necessari per blocchi di codice
\usepackage{listings}    
\usepackage{courier}
\usepackage[dvipsnames]{xcolor}   
\usepackage{pythonhighlight}   
\definecolor{verylightgray}{rgb}{.97,.97,.97}

\lstdefinelanguage{Solidity}{
	keywords=[1]{anonymous, assembly, assert, balance, break, call, callcode, case, catch, class, constant, continue, constructor, contract, debugger, default, delegatecall, delete, do, else, emit, event, experimental, export, external, false, finally, for, function, gas, if, implements, import, in, indexed, instanceof, interface, internal, is, length, library, log0, log1, log2, log3, log4, memory, modifier, new, payable, pragma, private, protected, public, pure, push, require, return, returns, revert, selfdestruct, send, solidity, storage, struct, suicide, super, switch, then, this, throw, transfer, true, try, typeof, using, value, view, while, with, addmod, ecrecover, keccak256, mulmod, ripemd160, sha256, sha3, virtual, override, abstract}, % generic keywords including crypto operations
	keywordstyle=[1]\color{blue}\bfseries,
	keywords=[2]{address, bool, byte, bytes, bytes1, bytes2, bytes3, bytes4, bytes5, bytes6, bytes7, bytes8, bytes9, bytes10, bytes11, bytes12, bytes13, bytes14, bytes15, bytes16, bytes17, bytes18, bytes19, bytes20, bytes21, bytes22, bytes23, bytes24, bytes25, bytes26, bytes27, bytes28, bytes29, bytes30, bytes31, bytes32, enum, int, int8, int16, int24, int32, int40, int48, int56, int64, int72, int80, int88, int96, int104, int112, int120, int128, int136, int144, int152, int160, int168, int176, int184, int192, int200, int208, int216, int224, int232, int240, int248, int256, mapping, string, uint, uint8, uint16, uint24, uint32, uint40, uint48, uint56, uint64, uint72, uint80, uint88, uint96, uint104, uint112, uint120, uint128, uint136, uint144, uint152, uint160, uint168, uint176, uint184, uint192, uint200, uint208, uint216, uint224, uint232, uint240, uint248, uint256, var, void, ether, finney, szabo, wei, days, hours, minutes, seconds, weeks, years},	% types; money and time units
	keywordstyle=[2]\color{teal}\bfseries,
	keywords=[3]{block, blockhash, coinbase, difficulty, gaslimit, number, timestamp, msg, data, gas, sender, sig, value, now, tx, gasprice, origin},	% environment variables
	keywordstyle=[3]\color{violet}\bfseries,
	identifierstyle=\color{black},
	sensitive=true,
	comment=[l]{//},
	morecomment=[s]{/*}{*/},
	commentstyle=\color{gray}\ttfamily,
	stringstyle=\color{red}\ttfamily,
	morestring=[b]',
	morestring=[b]"
}

\lstset{
	language=Solidity,
	backgroundcolor=\color{white},
	extendedchars=true,
	basicstyle=\footnotesize\ttfamily,
	showstringspaces=false,
	showspaces=false,
	basicstyle=\scriptsize\ttfamily,
	tabsize=2,
	breaklines=true,
	showtabs=false,
	captionpos=b,
	aboveskip=0pt,
	belowskip=0pt
}
\addto\captionsitalian{%
	\renewcommand{\lstlistingname}{Codice}}
%END - Necessari per blocchi di codice

%per grafici
\usepackage{float}
\usepackage{tikz}
\usepackage{pgfplots} 
\pgfplotsset{compat=1.16} 
\usetikzlibrary{datavisualization}
%END - Per grafici

\newlength{\defbaselineskip}
\setlength{\defbaselineskip}{\baselineskip}
\newcommand{\setlinespacing}[1]%
           {\setlength{\baselineskip}{#1 \defbaselineskip}}
\newcommand{\doublespacing}{\setlength{\baselineskip} {2.0 \defbaselineskip}}
\newcommand{\singlespacing}{\setlength{\baselineskip}{\defbaselineskip}}
\newcommand{\mycenterline}[1]{\vspace{.1cm}\newline\vspace{.1cm}\centerline{#1}}

% Bibliografia
\usepackage[sorting=none]{biblatex}
\addbibresource{components/bibliografica.bib}
%END - Bibliografia


\begin{document}
\selectlanguage{italian}
\begin{titlepage}
	
\centering

\includegraphics[scale=0.25]{components/images/university-logo}

\bigskip

\gdef\@phd@university{
	{\LARGE \bfseries Universit\`{a} degli Studi di Catania}\\ 
	{\large Dipartimento di Matematica e Informatica}\\ 
	{Corso di Laurea Triennale in Informatica}\\
	\bigskip}

\textsc{\@phd@university}\par

\hrule

\bigskip

\bigskip

\bigskip

\bigskip

\bigskip

\bigskip

{\itshape \large
	Alessio Tudisco\par
}

\bigskip

\bigskip

\bigskip

\bigskip

\bigskip

\bigskip

{\Large Riconoscimento di design pattern su blockchain tramite analisi statica del codice}\par

\bigskip

\bigskip

\bigskip

\bigskip

\bigskip

\bigskip

\begin{minipage}[b]{8 cm}
	\hrule
	
	\bigskip
	
	{\centering \scshape Relazione Progetto Finale\par}
	
	\bigskip
	
	\hrule
\end{minipage}

\bigskip

\bigskip

\bigskip

\bigskip

\bigskip

\bigskip

\begin{tabular}[t]{ccc}
	
	\textsc{} & \hspace{8cm} &\textsc{Relatore}\\
	& \hspace{8cm} & Prof. Emiliano Alessio Tramontana\\
	
\end{tabular}

\bigskip

\bigskip


%\begin{tabular}[t]{ccc}
%	
%	\textsc{} & \hspace{8cm} &\textsc{Correlatore}\\
%	& \hspace{8cm} & /\\
%	
%\end{tabular}

\bigskip

\bigskip

\bigskip

\bigskip

\bigskip

\bigskip

\bigskip

\bigskip

\bigskip

\bigskip

\bigskip

\bigskip

\hrule

\bigskip

{
	Anno Accademico 2021 - 2022\par
}
\end{titlepage}


\chaptertitlestyle{serifbig}
\pagestyle{serif}
\chapter*{Abstract}\label{abstract}
Blockchain technologies assume a key role in Web3, the next generation of the World Wide Web, whose goal is to create a decentralized and autonomous Internet in which users have greater control and ownership of their online data and activities. Within the Ethereum blockchain, automated digital contracts, called \textit{"smart-contracts"}, are executed when certain conditions are met. Smart-contracts are written in code by a developer and as such are susceptible to vulnerabilities and security problems. A coding error or vulnerability can allow an attacker to compromise the contract and cause extensive and irreparable damage. Therefore, it is important that smart-contracts are carefully written and tested to ensure that they are free of vulnerabilities and function as intended.\\
\\
The goal of this thesis is to document what \textit{design patterns} have been designed for Ethereum blockchain smart-contracts and to develop a software application capable of automatically analyzing a smart-contract in order to detect what \textit{design patterns} have been used.
\vspace{15pt}
\begin{center}
\large$\star\star\star$
\end{center}
\vspace{15pt}
Le tecnologie blockchain assumono un ruolo chiave nel Web3, la prossima generazione del World Wide Web, il cui obiettivo è creare un Internet decentralizzato e autonomo in cui gli utenti hanno maggiore controllo e proprietà dei propri dati e attività online. All'interno della blockchain Ethereum vengono eseguiti, quando determinate condizioni sono soddisfatte, dei contratti digitali automatizzati, denominati \textit{"smart-contract"}. Gli smart-contract sono scritti in codice da un programmatore e come tali sono suscettibili a vulnerabilità e problemi di sicurezza. Un errore di codifica o una vulnerabilità può permettere a un attaccante di compromettere il contratto e causare danni ingenti e irreparabili. Pertanto, è importante che gli smart-contract vengano scritti e testati con cura per garantire che siano privi di vulnerabilità e funzionino come previsto.\\
\\
L'obiettivo di questa tesi è documentare quali \textit{design pattern} siano stati ideati per gli smart-contract della blockchain Ethereum e sviluppare un applicativo software capace di analizzare automaticamente uno smart-contract al fine di rilevare quali \textit{design pattern} siano stati utilizzati.
 

\addtocontents{toc}{\protect\setcounter{tocdepth}{-1}}
\tableofcontents
\addtocontents{toc}{\protect\setcounter{tocdepth}{3}}

\begin{frontmatter}
\setlinespacing{1.4}
\pagenumbering{roman}
\pagenumbering{arabic} \setcounter{page}{1}
\end{frontmatter}


\setlinespacing{1.4}
\chapter{Introduzione e Contesto}
L'attuale generazione della tecnologia World Wide Web, denominata \textit{Web2}, è caratterizzata da una maggiore interattività e partecipazione degli utenti rispetto alla precedente generazione, detta \textit{Web1}. In genere, con il termine Web2, si denota la transizione dalla semplice navigazione di siti web all'utilizzo, da parte degli utenti della rete, di servizi e applicazioni interattive, come ad esempio: i social network, le piattaforme di streaming e gli e-commerce.

Questa seconda generazione è anche caratterizzata da un \textit{problema di centralizzazione}, il quale consiste nel fatto che molte delle informazioni e dei dati che circolano in rete sono gestiti da un numero ristretto di grandi aziende. Ciò può portare a problemi di privacy, sicurezza e libertà d'espressione, in quanto queste aziende possono utilizzare i dati degli utenti per fini commerciali o per influenzare l'accesso a determinate informazioni. Inoltre, la centralizzazione introduce problemi di dipendenza e vulnerabilità, in quanto un'unica entità ha il controllo sui dati e sui servizi forniti, la quale, se attaccata attraverso vulnerabilità, può essere soggetta al furto di una grande mole di dati sensibili.

La prossima generazione, denominata \textit{Web3}, punta a risolvere questo problema attraverso la decentralizzazione e la creazione di applicazioni e servizi che non dipendono da un'unica entità o autorità centrale, ma che sono gestiti da una rete di nodi distribuiti. In questo modo, si mira a creare una rete più equa, sicura e resiliente, dove gli utenti hanno il controllo dei propri dati e delle proprie informazioni.

Di seguito in questo capitolo vengono presentati i concetti fondamentali del Web3, il tema d'interesse della tesi, le modalità e il lavoro svolto per lo svolgimento quest'ultimo e la struttura generale della tesi.
\section{Le blockchain}
Le blockchain sono una tecnologia di registro distribuito che consente la creazione di una rete peer-to-peer senza la necessità di intermediari. Ciò significa che le transazioni e le informazioni possono essere scambiate direttamente tra gli utenti senza la necessità di un'autorità centrale. Ciò rende le blockchain adatte per la creazione di un Internet decentralizzato in cui gli utenti hanno maggiore sicurezza, trasparenza e autonomia.\\
Gli smart contract sono una forma di contratto digitale automatizzato che si esegue automaticamente quando determinate condizioni sono soddisfatte. Essi sono scritti in un linguaggio di programmazione che consente la creazione di regole e logiche specifiche che vengono eseguite su una blockchain.

Gli smart contract consentono la creazione di accordi automatici tra le parti in modo da ridurre i tempi di elaborazione e i costi associati alle transazioni tradizionali. Inoltre, essi possono essere utilizzati per creare applicazioni decentralizzate (dApps) che utilizzano la tecnologia blockchain per garantire la sicurezza e la trasparenza delle transazioni.

Per quanto riguarda la sicurezza, gli smart contract sono scritti in codice e come tali sono suscettibili a vulnerabilità e problemi di sicurezza. Un errore di codifica o una vulnerabilità può permettere a un attaccante di compromettere il contratto e causare danni irreparabili. Pertanto, è importante che gli smart contract vengano scritti e testati con cura per garantire che siano privi di vulnerabilità e funzionino come previsto.

\section{Struttura Tesi}
Di seguito vengono illustrati gli argomenti trattati in ogni capitolo presente:
\begin{itemize}
	\item Nel \textit{capitolo 2} vengono introdotti gli strumenti utilizzati per lo svolgimento del tema della tesi;
	\item Nel \textit{capitolo 3} vengono descritti i design pattern individuati nello studio dell'attuale divulgazione scientifica;
	\item Nel \textit{capitolo 4} viene presentato il software di analisi statica automatica sviluppato;
	\item Nel \textit{capitolo 5} vengono riportati i risultati dell'analisi automatica di alcuni smart-contract open-source;
	\item Nell'\textit{appendice} vengono riportati i codici di riferimento dei design pattern usati per lo sviluppo del software;
\end{itemize}


\begin{frontmatter}
\pagenumbering{arabic}
\chapter{Conclusioni}
Termina così lo sviluppo dell'applicativo software proposto nella tesi.\\
\newline
\textit{Solidity Design Pattern Analyzer} è un software \textit{open-source} rilasciato con licenza \textit{MIT}, il cui codice sorgente è liberalmente accessibile nel relativo repository su GitHub\cite{github-repo}.\\
\newline
L'applicativo software è in grado di eseguire le seguenti operazioni:
\begin{itemize}
	\item Rilevare, nei limiti linguistici e delle dipendenze utilizzate, tutti e ventidue i design pattern documentati nella tesi, i cui relativi descriptor sono inclusi nel codice sorgente, ed è possibile, mediante la combinazione di controlli generici, definire nuovi descriptor per riconoscere design pattern futuri;
	\item Descrivere uno smart-contract, ovvero estrarre le informazioni utili a creare un nuovo descriptor;
\end{itemize} 
Si conclude la tesi elencando un'ipotetica serie di sviluppi futuri:
\begin{itemize}
	\item Introdurre il supporto di dizionari di lingue diverse dall'inglese al fine di permettere l'analisi degli smart-contract le cui informazioni utili, come nomi di variabili di stato o di funzioni, corrispondono a parole di lingue straniere;
	\item Implementare un sistema di raggruppamento e interdipendenza dei controlli allo scopo di rilevare costrutti più articolati;
	\item Progettare un'interfaccia grafica per un utilizzo più user-friendly dell'applicativo;
\end{itemize}
 
 
\printbibliography[heading=bibintoc]
\end{frontmatter}
\end{document}
