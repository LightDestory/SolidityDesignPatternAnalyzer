\documentclass[12pt,oneside,a4paper,PisaPhdThesis, italian]{PhdThesis}
\usepackage[centertags]{amsmath}
\usepackage{framed}
\usepackage{libertine}
\usepackage{csquotes}
\usepackage[italian, english]{babel}
\usepackage[font=small,labelfont=bf,tableposition=top]{caption}
\usepackage{graphics}
\usepackage{graphicx}
\usepackage{color}
\usepackage{hyperref}
\usepackage{epsfig}
\usepackage{amsfonts}
\usepackage{amssymb}
\usepackage{amsthm}
\usepackage{rawfonts}
\usepackage{enumerate}
\usepackage{capt-of}
\usepackage{url}
\usepackage{xspace}
\usepackage{xthesis} 
\usepackage{xtocinc}

%Necessari per blocchi di codice
\usepackage{listings}    
\usepackage{courier}
\usepackage[dvipsnames]{xcolor}
\definecolor{commentgreen}{rgb}{0,0.45,0}
\definecolor{codepurple}{rgb}{0.58,0,0.82}
\definecolor{backcolour}{rgb}{0.95,0.95,0.92}
\definecolor{darkred}{rgb}{0.6,0.0,0.0}
\definecolor{lightblue}{rgb}{0.0,0.42,0.91}

\lstdefinelanguage{Solidity}{
	keywords=[1]{anonymous, assembly, assert, balance, break, call, callcode, case, catch, class, constant, continue, constructor, contract, debugger, default, delegatecall, delete, do, else, emit, event, experimental, export, external, false, finally, for, function, gas, if, implements, import, in, indexed, instanceof, interface, internal, is, length, library, log0, log1, log2, log3, log4, memory, modifier, new, payable, pragma, private, protected, public, pure, push, require, return, returns, revert, selfdestruct, send, solidity, storage, struct, suicide, super, switch, then, this, throw, transfer, true, try, typeof, using, value, view, while, with, addmod, ecrecover, keccak256, mulmod, ripemd160, sha256, sha3, virtual, override, abstract}, % generic keywords including crypto operations
	keywordstyle=\color{codepurple}\bfseries,
	keywords=[2]{address, bool, byte, bytes, bytes1, bytes2, bytes3, bytes4, bytes5, bytes6, bytes7, bytes8, bytes9, bytes10, bytes11, bytes12, bytes13, bytes14, bytes15, bytes16, bytes17, bytes18, bytes19, bytes20, bytes21, bytes22, bytes23, bytes24, bytes25, bytes26, bytes27, bytes28, bytes29, bytes30, bytes31, bytes32, enum, int, int8, int16, int24, int32, int40, int48, int56, int64, int72, int80, int88, int96, int104, int112, int120, int128, int136, int144, int152, int160, int168, int176, int184, int192, int200, int208, int216, int224, int232, int240, int248, int256, mapping, string, uint, uint8, uint16, uint24, uint32, uint40, uint48, uint56, uint64, uint72, uint80, uint88, uint96, uint104, uint112, uint120, uint128, uint136, uint144, uint152, uint160, uint168, uint176, uint184, uint192, uint200, uint208, uint216, uint224, uint232, uint240, uint248, uint256, var, void, ether, finney, szabo, wei, days, hours, minutes, seconds, weeks, years},	% types; money and time units
	keywordstyle=[2]\color{teal}\bfseries,
	keywords=[3]{block, blockhash, coinbase, difficulty, gaslimit, number, timestamp, msg, data, gas, sender, sig, value, now, tx, gasprice, origin},	% environment variables
	keywordstyle=[3]\color{lightblue}\bfseries,
	sensitive=true,
	comment=[l]{//},
	morecomment=[s]{/*}{*/},
	morestring=[b]',
	morestring=[b]"
}

\lstset{
	frame=lines,
	backgroundcolor=\color{backcolour},
	basicstyle=\scriptsize\ttfamily,
	breakatwhitespace=false,         
	breaklines=true,                 
	captionpos=b,                    
	keepspaces=true,                 
	numbers=left,  
	numberstyle=\tiny,                 
	numbersep=5pt,                  
	showspaces=false,                
	showstringspaces=false,
	showtabs=false,                  
	tabsize=2,
	framextopmargin=3pt,
	framexbottommargin=3pt,
	framexleftmargin=3pt,
	commentstyle=\color{commentgreen}\ttfamily,
	keywordstyle=\color{blue}\bfseries,
	identifierstyle=\color{black}\ttfamily,
	stringstyle=\color{darkred}\ttfamily,
}
\addto\captionsitalian{%
	\renewcommand{\lstlistingname}{Codice}}
%END - Necessari per blocchi di codice

%per grafici
\usepackage{float}
\usepackage{tikz}
\usepackage{pgfplots} 
\pgfplotsset{compat=1.16} 
\usetikzlibrary{datavisualization}
%END - Per grafici

\newlength{\defbaselineskip}
\setlength{\defbaselineskip}{\baselineskip}
\newcommand{\setlinespacing}[1]%
           {\setlength{\baselineskip}{#1 \defbaselineskip}}
\newcommand{\doublespacing}{\setlength{\baselineskip} {2.0 \defbaselineskip}}
\newcommand{\singlespacing}{\setlength{\baselineskip}{\defbaselineskip}}
\newcommand{\mycenterline}[1]{\vspace{.1cm}\newline\vspace{.1cm}\centerline{#1}}

% Bibliografia
\usepackage[sorting=none]{biblatex}
\addbibresource{components/bibliografica.bib}
%END - Bibliografia


\begin{document}
\selectlanguage{italian}
\begin{titlepage}
	
\centering

\includegraphics[scale=0.25]{components/images/university-logo}

\bigskip

\gdef\@phd@university{
	{\LARGE \bfseries Universit\`{a} degli Studi di Catania}\\ 
	{\large Dipartimento di Matematica e Informatica}\\ 
	{Corso di Laurea Triennale in Informatica}\\
	\bigskip}

\textsc{\@phd@university}\par

\hrule

\bigskip

\bigskip

\bigskip

\bigskip

\bigskip

\bigskip

{\itshape \large
	Alessio Tudisco\par
}

\bigskip

\bigskip

\bigskip

\bigskip

\bigskip

\bigskip

{\Large Riconoscimento di design pattern su blockchain tramite analisi statica del codice}\par

\bigskip

\bigskip

\bigskip

\bigskip

\bigskip

\bigskip

\begin{minipage}[b]{8 cm}
	\hrule
	
	\bigskip
	
	{\centering \scshape Relazione Progetto Finale\par}
	
	\bigskip
	
	\hrule
\end{minipage}

\bigskip

\bigskip

\bigskip

\bigskip

\bigskip

\bigskip

\begin{tabular}[t]{ccc}
	
	\textsc{} & \hspace{8cm} &\textsc{Relatore}\\
	& \hspace{8cm} & Prof. Emiliano Alessio Tramontana\\
	
\end{tabular}

\bigskip

\bigskip


%\begin{tabular}[t]{ccc}
%	
%	\textsc{} & \hspace{8cm} &\textsc{Correlatore}\\
%	& \hspace{8cm} & /\\
%	
%\end{tabular}

\bigskip

\bigskip

\bigskip

\bigskip

\bigskip

\bigskip

\bigskip

\bigskip

\bigskip

\bigskip

\bigskip

\bigskip

\hrule

\bigskip

{
	Anno Accademico 2021 - 2022\par
}
\end{titlepage}


\chaptertitlestyle{serifbig}
\pagestyle{serif}
\setlinespacing{1.1}
\chapter*{Abstract}\label{abstract}
Blockchain technologies assume a key role in Web3, the next generation of the World Wide Web, whose goal is to create a decentralized and autonomous Internet in which users have greater control and ownership of their online data and activities. Within the Ethereum blockchain, automated digital contracts, called \textit{"smart-contracts"}, are executed when certain conditions are met. Smart-contracts are written in code by a developer and as such are susceptible to vulnerabilities and security problems. A coding error or vulnerability can allow an attacker to compromise the contract and cause extensive and irreparable damage. Therefore, it is important that smart-contracts are carefully written and tested to ensure that they are free of vulnerabilities and function as intended.\\
\\
The goal of this thesis is to document what \textit{design patterns} have been designed for Ethereum blockchain smart-contracts and to develop a software application capable of automatically analyzing a smart-contract in order to detect what \textit{design patterns} have been used.
\vspace{15pt}
\begin{center}
\large$\star\star\star$
\end{center}
\vspace{15pt}
Le tecnologie blockchain assumono un ruolo chiave nel Web3, la prossima generazione del World Wide Web, il cui obiettivo è creare un Internet decentralizzato e autonomo in cui gli utenti hanno maggiore controllo e proprietà dei propri dati e attività online. All'interno della blockchain Ethereum vengono eseguiti, quando determinate condizioni sono soddisfatte, dei contratti digitali automatizzati, denominati \textit{"smart-contract"}. Gli smart-contract sono scritti in codice da un programmatore e come tali sono suscettibili a vulnerabilità e problemi di sicurezza. Un errore di codifica o una vulnerabilità può permettere a un attaccante di compromettere il contratto e causare danni ingenti e irreparabili. Pertanto, è importante che gli smart-contract vengano scritti e testati con cura per garantire che siano privi di vulnerabilità e funzionino come previsto.\\
\\
L'obiettivo di questa tesi è documentare quali \textit{design pattern} siano stati ideati per gli smart-contract della blockchain Ethereum e sviluppare un applicativo software capace di analizzare automaticamente uno smart-contract al fine di rilevare quali \textit{design pattern} siano stati utilizzati.
 


\begin{frontmatter}
\setlinespacing{1.1}
\pagenumbering{roman}
\addtocontents{toc}{\protect\setcounter{tocdepth}{-1}}
\tableofcontents
\addtocontents{toc}{\protect\setcounter{tocdepth}{3}}
\end{frontmatter}

\begin{frontmatter}\end{frontmatter}

\pagenumbering{arabic} \setcounter{page}{1}
\chapter{Introduzione e Contesto}
L'attuale generazione della tecnologia World Wide Web, denominata \textit{Web2}, è caratterizzata da una maggiore interattività e partecipazione degli utenti rispetto alla precedente generazione, detta \textit{Web1}. In genere, con il termine Web2, si denota la transizione dalla semplice navigazione di siti web all'utilizzo, da parte degli utenti della rete, di servizi e applicazioni interattive, come ad esempio: i social network, le piattaforme di streaming e gli e-commerce.

Questa seconda generazione è anche caratterizzata da un \textit{problema di centralizzazione}, il quale consiste nel fatto che molte delle informazioni e dei dati che circolano in rete sono gestiti da un numero ristretto di grandi aziende. Ciò può portare a problemi di privacy, sicurezza e libertà d'espressione, in quanto queste aziende possono utilizzare i dati degli utenti per fini commerciali o per influenzare l'accesso a determinate informazioni. Inoltre, la centralizzazione introduce problemi di dipendenza e vulnerabilità, in quanto un'unica entità ha il controllo sui dati e sui servizi forniti, la quale, se attaccata attraverso vulnerabilità, può essere soggetta al furto di una grande mole di dati sensibili.

La prossima generazione, denominata \textit{Web3}, punta a risolvere questo problema attraverso la decentralizzazione e la creazione di applicazioni e servizi che non dipendono da un'unica entità o autorità centrale, ma che sono gestiti da una rete di nodi distribuiti. In questo modo, si mira a creare una rete più equa, sicura e resiliente, dove gli utenti hanno il controllo dei propri dati e delle proprie informazioni.

Di seguito in questo capitolo vengono presentati i concetti fondamentali del Web3, il tema d'interesse della tesi, le modalità e il lavoro svolto per lo svolgimento quest'ultimo e la struttura generale della tesi.
\section{Le blockchain}
Le blockchain sono una tecnologia di registro distribuito che consente la creazione di una rete peer-to-peer senza la necessità di intermediari. Ciò significa che le transazioni e le informazioni possono essere scambiate direttamente tra gli utenti senza la necessità di un'autorità centrale. Ciò rende le blockchain adatte per la creazione di un Internet decentralizzato in cui gli utenti hanno maggiore sicurezza, trasparenza e autonomia.\\
Gli smart contract sono una forma di contratto digitale automatizzato che si esegue automaticamente quando determinate condizioni sono soddisfatte. Essi sono scritti in un linguaggio di programmazione che consente la creazione di regole e logiche specifiche che vengono eseguite su una blockchain.

Gli smart contract consentono la creazione di accordi automatici tra le parti in modo da ridurre i tempi di elaborazione e i costi associati alle transazioni tradizionali. Inoltre, essi possono essere utilizzati per creare applicazioni decentralizzate (dApps) che utilizzano la tecnologia blockchain per garantire la sicurezza e la trasparenza delle transazioni.

Per quanto riguarda la sicurezza, gli smart contract sono scritti in codice e come tali sono suscettibili a vulnerabilità e problemi di sicurezza. Un errore di codifica o una vulnerabilità può permettere a un attaccante di compromettere il contratto e causare danni irreparabili. Pertanto, è importante che gli smart contract vengano scritti e testati con cura per garantire che siano privi di vulnerabilità e funzionino come previsto.

\section{Struttura Tesi}
Di seguito vengono illustrati gli argomenti trattati in ogni capitolo presente:
\begin{itemize}
	\item Nel \textit{capitolo 2} vengono introdotti gli strumenti utilizzati per lo svolgimento del tema della tesi;
	\item Nel \textit{capitolo 3} vengono descritti i design pattern individuati nello studio dell'attuale divulgazione scientifica;
	\item Nel \textit{capitolo 4} viene presentato il software di analisi statica automatica sviluppato;
	\item Nel \textit{capitolo 5} vengono riportati i risultati dell'analisi automatica di alcuni smart-contract open-source;
	\item Nell'\textit{appendice} vengono riportati i codici di riferimento dei design pattern usati per lo sviluppo del software;
\end{itemize}
\chapter{Strumenti Utilizzati}
L'applicativo software sviluppato per automatizzare il processo di analisi statica del codice sorgente di uno smart-contract, scritto in Solidity, legge il codice sorgente, fornito in input, rileva l'utilizzo di specifici design pattern e infine fornisce, in output, le informazioni ottenute all'utente finale.\par
Per lo sviluppo di tale applicativo si è fatto affidamento a numerose tecnologie.

{\section{Python}

\begin{wrapfigure}{r}{0.20\textwidth}
	\centering
	\includegraphics[scale=0.1]{python-logo}
\end{wrapfigure}
Python\cite{python} è un linguaggio di programmazione ad alto livello con una sintassi semplice e intuitiva. È un linguaggio interpretato con tipizzazione dinamica debole, ovvero la tipizzazione delle variabili avviene durante l'esecuzione del codice. Questo rende Python molto flessibile e facile da utilizzare.\par
Lo sviluppo di Python ha avuto inizio negli anni '90 da Guido van Rossum e si è rapidamente affermato come uno dei linguaggi di programmazione più popolari, grazie sopratutto alla sua versatilità in svariate campi, tra cui web development, sviluppo di software scientifico, automazione, machine learning, statistica e molto altro ancora.\par
Uno dei punti di forza di Python è la sua estesa libreria standard, che include molte funzioni già implementate per molte attività comuni, come l'elaborazione di stringhe, la gestione dei file, la connessione a database, il parsing di XML e JSON. Ad affiancare la libreria standard ci sono anche numerosi sviluppatori che hanno creato molti framework e librerie esterne per molte applicazioni specifiche.\par
In sintesi, Python è un linguaggio di programmazione estremamente potente e versatile che rende facile sviluppare software in molti ambiti diversi. La sua sintassi semplice e intuitiva, la vasta libreria standard e la forte comunità di sviluppatori lo rendono una scelta popolare per molti programmatori.

\subsection{Python-Solidity-Parser}
Alla base della tecnica di analisi statica vi è una componente software denominata \textit{"parser"}. Un parser analizza un input, spesso una stringa o un file testuale, e lo trasforma in una struttura dati più significativa, come un albero sintattico.\par
La struttura dati prodotta da un parser può essere facilmente manipolata o elaborata da altre componenti software, le quali potranno effettuare un'analisi semantica dei dati e pendere decisioni basandosi su di essa.\\ \newline
Il processo di parsing consiste di solito di diverse fasi, tra cui:
\begin{itemize}
	\item \textit{Analisi lessicale}: in cui si analizza l'input carattere per carattere e lo divide in token, ovvero unità più piccole e significative;
	\item \textit{Analisi sintattica}: in cui si analizza la sequenza di token generata dalla fase di analisi lessicale e si verifica che seguano la grammatica, spesso generata con ANTLR\cite{ANTLR}, del linguaggio di input. Se viene rilevato un errore sintattico, il parser solitamente lo segnala e interrompe il processo di parsing;
	\item \textit{Costruzione dell'albero sintattico}: in cui si costruisce un albero sintattico, una struttura dati che rappresenta la struttura sintattica dell'input;
\end{itemize}
Per quanto riguarda Solidity e Python, esiste un parser sperimentale open-source sviluppato in Python denominato \textit{"python-solidity-parser"}\cite{python-solidity-parser}. Non offre tutte le funzionalità tipiche di un parser e risulta parecchio acerbo, ma rappresenta una componente fondamentale per lo sviluppo dell'applicativo della tesi.
}
{\section{JSON}
	
	\begin{wrapfigure}{l}{0.20\textwidth}
		\centering
		\includegraphics[scale=0.09]{json-logo}
	\end{wrapfigure}
	JSON (\textit{JavaScript Object Notation}) è un formato di testo semplice e leggero per lo scambio di dati tra sistemi informatici. JSON è stato progettato per essere facile da leggere e scrivere sia per le macchine sia per gli esseri umani, può essere facilmente manipolato ed elaborato da molti linguaggi di programmazione come Python, Java, JavaScript e molti altri. Per questo motivo è diventato uno dei formati di dati più diffusi per l'integrazione tra applicazioni e il trasporto di dati tramite internet. \par
	I dati sono rappresentati come coppie chiave-valore, dove la chiave è una stringa e il valore può essere una stringa, un numero, un booleano, un array o un altro oggetto JSON. Gli oggetti JSON sono delimitati da parentesi graffe "\{\}" e i membri dell'oggetto sono separati da virgole. Gli array JSON, invece, sono delimitati da parentesi quadre "[]" e i membri dell'array sono separati da virgole.
	
	{\begin{lstlisting}[language=json, caption={Esempio di oggetto JSON}]
{
	"nome": "Mario",
	"cognome": "Rossi",
	"eta": 30,
	"genere": "M",
	"indirizzo": {
		"via": "Via Roma",
		"citta": "Roma",
		"paese": "Italia"
	},
	"interessi": ["cinema", "sport", "lettura"]
}\end{lstlisting}}
	
	\subsection{JSON Schema Draft-07}
	JSON Schema\cite{JSON-Schema} è un linguaggio formale per descrivere la struttura di un documento JSON. Permette di definire un modello di dati, un insieme di regole e restrizioni, con cui convalidare una struttura dati JSON e verificare che sia ben formattata, al fine di prevenire errori e garantire la qualità dei dati. Tramite un modello di dati si potrebbe richiedere la presenza obbligatoria di determinati campi, la lunghezza minima o massima di una stringa, o il tipo di dati che un determinato campo può contenere.\par
	Il draft-07\cite{JSON-Schema-Draft-07} è la settima specifica del formato JSON Schema, viene utilizzata questa versione poiché la libreria di validazione per Python ha un supporto parziale per le versioni successive.
{\begin{lstlisting}[language=json, caption={Esempio di modello di dati JSON-Schema}]
{
	"$schema": "http://json-schema.org/draft-07/schema#",
	"type": "object",
	"properties": {
		"nome": {
			"type": "string"
		},
		
		"cognome": {
			"type": "string"
		},
		"eta": {
			"type": "integer"
		},
		"genere": {
			"type": "string"
		},	
	},
	"required": ["nome", "cognome", "eta"]
}\end{lstlisting}}
}
\chapter{Classificazione dei Design Pattern per Smart-Contract}
I design pattern adottati nello sviluppo di smart-contract possono essere classificati in diversi tipi\cite[alcuni tipi]{9089272}\cite{9050163}, ognuno dei quali caratterizza un aspetto specifico della lifecycle del contratto:

\begin{itemize}
	\item \textit{Authorization}: relativi alla gestione dell'accesso alle funzionalità del contratto;
	\item \textit{Behavioral}: relativi a meccanismi di supporto per il corretto svolgimento delle funzionalità del contratto;
	\item \textit{Gas Economic}: relativi a meccanismi per ridurre il consumo di \textit{gas} durante l'esecuzione delle funzionalità del contratto;
	\item \textit{Lifecycle}: relativi alla a meccanismi per la creazione e la distruzione del contratto;
	\item \textit{Maintenance}: relativi a meccanismi di supporto per la manutenzione del contratto;
	\item \textit{Security}: relativi a meccanismi per la mitigazione di vulnerabilità di sicurezza note;
\end{itemize}

{\section{Authorization Design Pattern}
La blockchain Ethereum non implementa meccanismi di autenticazione o di permessi per consentire l'accesso alle funzionalità di un contratto solo nel caso in cui vengano soddisfatte determinate condizioni. Per la natura della blockchain, ogni funzione definita con visibilità pubblica può essere invocata da qualsiasi utente.\par
Gli \textit{Authorization Design Pattern} propongono meccanismi atti a limitare l'esecuzione delle funzionalità di un contratto.\par
Si individuano i seguenti design pattern: \textit{Access Restriction} e \textit{Ownership}.

{\subsection{Access Restriction Pattern}
	L’\textit{Access Restriction} pattern propone un meccanismo per introdurre dei controlli di prerequisiti nella logica del contratto.\par
	Il meccanismo proposto si basa sul concetto di \textit{function modifiers} presente in Solidity, che ci permette di definire controlli che possono essere eseguiti prima o dopo l'esecuzione del corpo della funzione a cui viene assegnato il modifier.
	\begin{table}[H]
		\centering
		\begin{tikzpicture}
			\node (table) [inner sep=0pt] {
				\def\arraystretch{1.5}
				\begin{tabular}{p{0.30\linewidth} | p{0.65\linewidth}}
					\textbf{Problema} & {Qualsiasi utente può richiamare qualunque funzione definita con visibilità pubblica di un contratto, ma in certi use-case potrebbe essere necessario definire dei prerequisiti o condizioni ed eseguire la funzionalità solo al verificarsi di essi.} \\ \hline
					\textbf{Soluzione} & {Definire una serie di \textit{modifier} generalmente applicabili che controllano i prerequisiti desiderati e inserirli nella definizione delle funzioni interessate.} \\ \hline
					\textbf{Riconoscimento} & {Individuare  all'interno del contratto l'utilizzo o la definizione dei \textit{modifier}.} \\ \hline
					\textbf{Snippet di Codice\newline(Versione Semplificata)} & Il codice sorgente di riferimento è consultabile nell'\hyperref[appendix:access_restriction]{Appendice}.  \\ \hline
					\textbf{Riferimenti} & \cite[Access Restriction]{maxwoe} \cite[Access Restriction]{cjgdev} \cite[Access Restriction]{fravoll} \\
				\end{tabular}
			};
			\draw [rounded corners=.5em] (table.north west) rectangle (table.south east);
		\end{tikzpicture}
		\caption{Specifiche Access Restriction Pattern}
	\end{table}
}
\newpage
{\subsection{Ownership Pattern}
L’\textit{Ownership} pattern propone un meccanismo per riservare al proprietario di uno smart-contract l’esecuzione di funzionalità critiche, a cui l’utente finale non deve avere accesso.\par
Opzionalmente possono essere implementati anche meccanismi di supporto per eseguire l’operazione di trasferimento del titolo di proprietario.
Può essere considerato un caso specifico di Access Restriction.
\begin{table}[H]
	\centering
	\begin{tikzpicture}
		\node (table) [inner sep=0pt] {
			\def\arraystretch{1.5}
			\begin{tabular}{p{0.30\linewidth} | p{0.65\linewidth}}
				\textbf{Problema} & {Qualsiasi utente può richiamare qualunque funzione definita con visibilità pubblica di un contratto, ma in certi use-case potrebbe essere necessario riservare l’esecuzione di specifiche funzioni al solo proprietario.} \\ \hline
				\textbf{Soluzione} & {Memorizzare in una variabile di stato del contratto l'indirizzo del proprietario e limitare l'esecuzione delle funzioni in base all'indirizzo del chiamante.} \\ \hline
				\textbf{Riconoscimento} & {Individuare un confronto fra \textit{msg.sender} e la variabile di stato contenente l’indirizzo del proprietario.\par Il confronto può essere implementato all'interno della funzione di interesse attraverso un costrutto \textit{if-else}, \textit{modifier} o la funzione \textit\mbox{\textit{require()}}.} \\ \hline
				\textbf{Snippet di Codice\newline(Versione Semplificata)} & Il codice sorgente di riferimento è consultabile nell'\hyperref[appendix:ownership]{Appendice}.  \\ \hline
				\textbf{Riferimenti} & \cite[Ownership]{maxwoe} \cite[Ownership]{cjgdev} \cite[Incluso in Access Restriction]{fravoll} \\
			\end{tabular}
		};
		\draw [rounded corners=.5em] (table.north west) rectangle (table.south east);
	\end{tikzpicture}
	\caption{Specifiche Ownership Pattern}
\end{table}
}
}

{\section{Behavioral Design Pattern}
I Behavioral design pattern propongono funzionalità e meccanismi di supporto per lo svolgimento delle operazioni del contratto.\par
Si individuano i seguenti design pattern: \textit{Commit \& Reveal}, \textit{Guard Check}, \textit{Oracle}, \textit{Pull Payment (Pull over Push)}, \textit{Randomness} e \textit{State Machine Pattern}.
{\subsection{Commit and Reveal Pattern}
	Il \textit{Commit and Reveal} pattern propone un meccanismo per consentire agli utenti di uno smart-contract di attenersi a un valore tenendolo nascosto agli altri con la possibilità di rivelarlo in seguito.
	\begin{table}[H]
		\centering
		\begin{tikzpicture}
			\node (table) [inner sep=0pt] {
				\def\arraystretch{1.5}
				\begin{tabular}{p{0.30\linewidth} | p{0.65\linewidth}}
					\textbf{Problema} & {Essendo Ethereum una blockchain pubblica, tutti i dati e tutte le transazioni sono visibili pubblicamente. Si può facilmente immaginare un caso d’uso, come una scommessa basata sul gioco carta-forbici-sasso, in cui le interazioni con lo smart-contract, particolarmente i valori dei parametri inviati, debbano essere trattate in modo confidenziale.} \\ \hline
					\textbf{Soluzione} & {Implementare uno schema di commitment. Uno schema di commitment è un algoritmo crittografico utilizzato per consentire a qualcuno di impegnarsi su un valore tenendolo nascosto ad altri con la possibilità di rivelarlo in seguito. I valori in uno schema di commitment sono vincolanti, ovvero nessuno può cambiarli una volta impegnati. Lo schema prevede due fasi: una fase di commit in cui viene scelto e specificato un valore e una fase di reveal in cui il valore viene rivelato e utilizzato.} \\ \hline
					\textbf{Riconoscimento} & {Data la complessità e la libertà implementativa, il riconoscimento di questo design pattern si basa sulla ricerca di definizione e uso di funzioni denominate \textit{commit} e \textit{reveal} e sulla definizione degli eventi \textit{LogCommit} e \textit{LogReveal}. Si potrebbe affinare il riconoscimento tentanto di rilevare una struct di commit.} \\ \hline
					\textbf{Snippet di Codice\newline(Versione Semplificata)} & Il codice sorgente di riferimento è consultabile nell'\hyperref[appendix:commit_and_reveal]{Appendice}.  \\ \hline
					\textbf{Riferimenti} & \cite[CommitReveal]{maxwoe} \\
				\end{tabular}
			};
			\draw [rounded corners=.5em] (table.north west) rectangle (table.south east);
		\end{tikzpicture}
		\caption{Specifiche Commit and Reveal Pattern}
	\end{table}
}
{\subsection{Guard Check}
	Il \textit{Guard Check} pattern propone un meccanismo per assicurarsi che il 
	comportamento dello smart-contract sia quello previsto, che i parametri di input siano validi e che in caso di errore lo stato interno del contratto venga ripristinato al momento precedente l'esecuzione della funzione.\par
	Potremmo immaginare questo design pattern come una implementazione di numerosi test eseguiti a runtime per valutare lo stato d’esecuzione.
	\begin{table}[H]
		\centering
		\begin{tikzpicture}
			\node (table) [inner sep=0pt] {
				\def\arraystretch{1.5}
				\begin{tabular}{p{0.30\linewidth} | p{0.65\linewidth}}
					\textbf{Problema} & {Nella blockchain Ethereum non vi sono regolatori o mediatori, ma vi è bisogno di protezioni o controlli per assicurare che la logica degli smart-contract funzioni come specificato. Uno smart-contract dovrebbe verificare tutti i prerequisiti della funzionalità richiesta e procedere solo se tutto è come previsto. In caso di errori, il contratto dovrebbe ripristinare tutte le modifiche apportate al suo stato.} \\ \hline
					\textbf{Soluzione} & {Per ottenere questi comportamenti, Solidity sfrutta il modo in cui l'EVM gestisce gli errori: per mantenere l'atomicità, tutte le modifiche effettuate vengono annullate e l'intera transazione viene invalidata. Per innescare gli errori, e quindi ripristinare lo stato interno del contratto, Solidity utilizza una serie di eccezioni, ognuna avente un utilizzo specifico:
					\begin{itemize}
						\item \textit{assert(condition)}: usato solo per verificare la presenza di errori interni e per controllare gli invarianti (asserzioni sempre vere). Una caratteristica importante è che rimborsa tutto il gas che non è stato consumato nel momento in cui viene lanciata l'eccezione;
						\item \textit{require(condition)}: usato solo per garantire condizioni valide, come gli input o le variabili di stato del contratto, o per convalidare i valori di ritorno da chiamate a contratti esterni. Una caratteristica importante è che consuma tutto il gas incluso nella transazione;
						\item \textit{revert()}: usato per ripristinare le modifiche apportate allo stato, viene normalmente lanciato automaticamente in caso di fallimento da \textit{require()} o \textit{assert()} ma potrebbe essere utilizzato all’interno di un controllo fatto da un costrutto \textit{if};
					\end{itemize}} \\ \hline
					\textbf{Riconoscimento} & {Individuare l'utilizzo di una delle eccezioni: \textit{assert(condition)}, \textit{require(condition)} o \textit{revert()}.} \\ \hline
					\textbf{Snippet di Codice\newline(Versione Semplificata)} & Il codice sorgente di riferimento è consultabile nell'\hyperref[appendix:guardcheck]{Appendice}.  \\ \hline
					\textbf{Riferimenti} & \cite[Guard Check]{fravoll} \\
				\end{tabular}
			};
			\draw [rounded corners=.5em] (table.north west) rectangle (table.south east);
		\end{tikzpicture}
		\caption{Specifiche Guard Check Pattern}
	\end{table}
}
{\subsection{Oracle Pattern}
	L’\textit{Oracle} pattern propone un meccanismo per permettere a un contratto di ottenere informazioni dal mondo esterno, fuori dalla blockchain, necessarie al corretto funzionamento del contratto e non presenti all'interno della rete Ethereum.
	\begin{table}[H]
		\centering
		\begin{tikzpicture}
			\node (table) [inner sep=0pt] {
				\def\arraystretch{1.5}
				\begin{tabular}{p{0.30\linewidth} | p{0.65\linewidth}}
					\textbf{Problema} & {Si immagini un caso d’uso in cui lo smart-contract necessiti di informazioni presenti al di fuori della blockchain, come ad esempio l'attuale cambio di una specifica valuta. Gli smart-contract Ethereum non possono acquisirle direttamente poiché non possono contattare il mondo esterno, al contrario si affidano al mondo esterno che immette tali informazioni nella rete.} \\ \hline
					\textbf{Soluzione} & {Richiedere i dati esterni attraverso un servizio \textit{oracolo}, un mediatore collegato al mondo esterno che assume il ruolo di provider di dati.} \\ \hline
					\textbf{Riconoscimento} & {Data la complessità e la libertà implementativa, il riconoscimento di questo design pattern si basa sulla ricerca di definizione e uso di funzioni denominate \textit{query} e \textit{reply} e sulla definizione di eventi relativi alle richieste. Si potrebbe affinare il riconoscimento tentanto di rilevare una struct di richieste. } \\ \hline
					\textbf{Snippet di Codice\newline(Versione Semplificata)} & Il codice sorgente di riferimento è consultabile nell'\hyperref[appendix:oracle]{Appendice}  \\ \hline
					\textbf{Riferimenti} & \cite[Oracle]{maxwoe} \cite[Variante basata su Oraclize]{fravoll} \\
				\end{tabular}
			};
			\draw [rounded corners=.5em] (table.north west) rectangle (table.south east);
		\end{tikzpicture}
		\caption{Specifiche Oracle Pattern}
	\end{table}
}
\newpage
{\subsection{Full Payment (Pull Over Push) Pattern}
	Il \textit{Full Payment (Pull Over Push)} pattern propone un meccanismo per eseguire 
	pagamenti (trasferimenti di criptovaluta) in modo sicuro.
	\begin{table}[H]
		\centering
		\begin{tikzpicture}
			\node (table) [inner sep=0pt] {
				\def\arraystretch{1.5}
				\begin{tabular}{p{0.30\linewidth} | p{0.65\linewidth}}
					\textbf{Problema} & {L'invio di \textit{ether} comporta una chiamata all'entità ricevente. Tale chiamata esterna potrebbe fallire per diversi motivi: per esempio se l'indirizzo ricevente è un contratto, potrebbe essere implementata una funzione di \textit{fallback()} che lancia semplicemente un'eccezione, una volta chiamata. Un altro causa di fallimento potrebbe essere l'esaurimento del “gas” contenuto nella transazione.} \\ \hline
					\textbf{Soluzione} & {Mai non affidarsi al fatto che le chiamate esterne vengano eseguite senza lanciare un errore: potremmo dire che è responsabilità del destinatario assicurarsi di essere in grado di ricevere il denaro. Per questo motivo la situazione ideale è far sì che sia il destinatario a innescare il trasferimento.} \\ \hline
					\textbf{Riconoscimento} & {Individuare un trasferimento di \textit{ether} ad opera del destinatario, ovvero una chiamata a funzione del tipo: \textit{msg.sender.transfer(amount)}.} \\ \hline
					\textbf{Snippet di Codice\newline(Versione Semplificata)} & Il codice sorgente di riferimento è consultabile nell'\hyperref[appendix:pull_over_push]{Appendice}  \\ \hline
					\textbf{Riferimenti} & \cite[SendingFunds]{maxwoe} \cite[Pull Over Push]{fravoll} \\
				\end{tabular}
			};
			\draw [rounded corners=.5em] (table.north west) rectangle (table.south east);
		\end{tikzpicture}
		\caption{Specifiche Full Payment (Pull Over Push) Pattern}
	\end{table}
}
{\subsection{Randomness Pattern}
	Il \textit{Randomness} pattern propone un meccanismo per consentire a uno smart-contract di generare un numero casuale, appartenente a un intervallo predefinito, in un ambiente deterministico come la blockchain.
	\begin{table}[H]
		\centering
		\begin{tikzpicture}
			\node (table) [inner sep=0pt] {
				\def\arraystretch{1.5}
				\begin{tabular}{p{0.30\linewidth} | p{0.65\linewidth}}
					\textbf{Problema} & {La casualità nei sistemi informatici, e soprattutto in Ethereum, è notoriamente difficile da ottenere. Per quanto riguarda Ethereum, la rete è una macchina di Turing deterministica, senza alcuna casualità intrinseca ma, nonostante ciò, la necessità di casualità è assai elevata, si immagini un gioco la cui vincita si basa su un fattore di casualità. Un’altra problematica per la casualità è la natura pubblica di una blockchain: lo stato interno di un contratto, così come l'intera storia di una blockchain, è visibile al pubblico. Pertanto, è difficile trovare una fonte sicura di entropia.} \\ \hline
					\textbf{Soluzione} & {Sono state ideate diverse soluzioni per superare queste limitazioni:
					\begin{itemize}
						\item \textit{Block Hash PRNG}: l'hash del blocco viene usato come sorgente di casualità;
						\item \textit{Oracle PRNG}: un oracolo assume il ruolo di provider di casualità;
						\item \textit{Collaborative PRNG}: una generazione collaborativa di casualità dentro la blockchain;
					\end{itemize}
					} \\ \hline
					\textbf{Riconoscimento} & {Data la complessità e la libertà implementativa delle varianti basate sull'oracolo e la generazione collaborativa, si riconosce la variane più semplice basata sull'hash del blocco. L'hash del blocco può essere usato in diversi modi:
					\begin{itemize}
						\item \textit{uint(blockhash(block.number-1))}: non utilizzabile per scommesse in quanto facilmente determinabile;
						\item \textit{uint(keccak256(abi.encodePacked(blockhash( block.number-1), seed)))}: il seed viene generato dall'utente e fornito come parametro;
						\item \textit{uint(keccak256(abi.encodePacked(block.timestamp, msg.sender, block.difficulty)))}
					\end{itemize}
					} \\ \hline
					\textbf{Snippet di Codice\newline(Versione Semplificata)} & Il codice sorgente di riferimento è consultabile nell'\hyperref[appendix:randomness]{Appendice}  \\ \hline
					\textbf{Riferimenti} & \cite[Randomness]{fravoll} \\
				\end{tabular}
			};
			\draw [rounded corners=.5em] (table.north west) rectangle (table.south east);
		\end{tikzpicture}
		\caption{Specifiche Randomness Pattern}
	\end{table}
}
{\subsection{State Machine Pattern}
	Lo \textit{State Machine} pattern propone un meccanismo per consentire a uno smart-contract di passare, durante la sua esecuzione, attraverso diverse fasi con diverse funzionalità corrispondenti esposte.
	\begin{table}[H]
		\centering
		\begin{tikzpicture}
			\node (table) [inner sep=0pt] {
				\def\arraystretch{1.5}
				\begin{tabular}{p{0.30\linewidth} | p{0.65\linewidth}}
					\textbf{Problema} & {Consideriamo un contratto che deve passare da uno stato iniziale, attraverso diversi stati intermedi, allo stato finale nel corso della sua attività. In ognuno di questi stati il contratto deve comportarsi in modo diverso e fornire diverse funzionalità ai suoi utenti. Il comportamento descritto può essere osservato in una moltitudine di casi d'uso: aste, gioco d'azzardo, crowdfunding, ect...} \\ \hline
					\textbf{Soluzione} & {Implementare una macchina a stati per modellare e rappresentare le diverse fasi comportamentali del contratto e le loro transizioni.} \\ \hline
					\textbf{Riconoscimento} & {Data la complessità e la libertà implementativa, il riconoscimento di questo design pattern si basa sulla ricerca di definizione e uso di un \textit{modifier} relativo agli stati e la definizione di un \textit{enum} di stati.} \\ \hline
					\textbf{Snippet di Codice\newline(Versione Semplificata)} & Il codice sorgente di riferimento è consultabile nell'\hyperref[appendix:state_machine]{Appendice}  \\ \hline
					\textbf{Riferimenti} & \cite[StateMachine]{maxwoe} \\
				\end{tabular}
			};
			\draw [rounded corners=.5em] (table.north west) rectangle (table.south east);
		\end{tikzpicture}
		\caption{Specifiche State Machine Pattern}
	\end{table}
}
}

{\section{Gas Economic Design Pattern}
	I Gas Economic design pattern propongono meccanismi considerabili come \textit{best practice} per ridurre il consumo di gas durante l’esecuzione delle funzionalità dello smart-contract. \par
	Le operazioni con un consumo di gas altamente variabile e imprevedibile costituiscono un problema per i contratti, in quanto comportano il rischio che le transazioni rimangano senza gas e si ottengano di conseguenza dei comportamenti indesiderati. Pertanto, è auspicabile un fabbisogno di gas basso, stabile e prevedibile.\par
	Si individuano i seguenti design pattern: \textit{Memory Array Building}, \textit{String Equality Comparison} e \textit{Tight Variable Packing}.
	
	{\subsection{Memory Array Building Pattern}
		Il \textit{Memory Array Building} pattern propone un meccanismo per aggregare e recuperare i dati dallo storage di un contratto in modo efficiente dal punto di vista del consumo di gas.\par	L'interazione con lo storage di un contratto sulla blockchain è una delle operazioni più costose dell'EVM. Pertanto, è necessario memorizzare solo i dati necessari ed evitare, se possibile, la ridondanza. Nei sistemi tradizionali l'unico costo rilevante delle query in uno storage è il tempo, mentre in Ethereum anche semplici interrogazioni possono costare una quantità sostanziale di gas, che ha un riscontro monetario diretto.\par
		Si potrebbe mitigare il costo del gas relativo alla lettura di un dato definendo la variabile contenitore come pubblica, generando così un \textit{getter} in background che permette di accedere gratuitamente al valore della variabile, ma ciò non è sempre applicabile.
		\begin{table}[H]
			\centering
			\begin{tikzpicture}
				\node (table) [inner sep=0pt] {
					\def\arraystretch{1.5}
					\begin{tabular}{p{0.30\linewidth} | p{0.65\linewidth}}
						\textbf{Problema} & {Si immagini un caso d’uso in cui si voglia aggregare dati provenienti da diverse fonti, ciò richiederebbe una grande quantità di letture dallo storage e sarebbe quindi particolarmente costoso.} \\ \hline
						\textbf{Soluzione} & {Sfruttare il modifier \textit{view} di Solidity, che consente di leggere e aggregare i dati dallo \textit{storage} del contratto senza alcun costo associato. Le funzioni dichiarate con il modifier \textit{view} non sono autorizzate a scrivere nello \textit{storage} e quindi non modificano lo stato della blockchain. Poiché lo stato della blockchain rimane invariato, non è necessario trasmettere una transazione nella rete. Nessuna transazione significa nessun consumo di gas, il che rende gratuita la chiamata della funzione, purché sia chiamata esternamente e non da un altro contratto. In tal caso, sarebbe necessaria una transazione e verrebbe consumato del gas. Una funzione con tali caratteristiche permette di eseguire un lettura da diverse fonti, aggregare e ritornare i dati in un array definito in \textit{memory}, anziché nello \textit{storage}.} \\ \hline
						\textbf{Riconoscimento} & {Individuare la definizione di una funzione avente modifier \textit{view} che ritorni un array con locazione \textit{memory}.} \\ \hline
						\textbf{Snippet di Codice\newline(Versione Semplificata)} & Il codice sorgente di riferimento è consultabile nell'\hyperref[appendix:memory_array_building]{Appendice}.  \\ \hline
						\textbf{Riferimenti} & \cite[Memory Array Building]{fravoll} \cite[Limit Storage]{9050163} \\
					\end{tabular}
				};
				\draw [rounded corners=.5em] (table.north west) rectangle (table.south east);
			\end{tikzpicture}
			\caption{Specifiche Memory Array Building Pattern}
		\end{table}
	}
	{\subsection{String Equality Comparison Pattern}
		Lo \textit{String Equality Comparison} pattern propone un meccanismo per poter verificare l’uguaglianza di due stringhe in un modo che minimizzi il consumo medio del gas per un gran numero di input diversi.
		\begin{table}[H]
			\centering
			\begin{tikzpicture}
				\node (table) [inner sep=0pt] {
					\def\arraystretch{1.5}
					\begin{tabular}{p{0.30\linewidth} | p{0.65\linewidth}}
						\textbf{Problema} & {Confrontare le stringhe in altri linguaggi di programmazione è un compito banale, vi sono metodi o pacchetti integrati che possono verificare l'uguaglianza di due input con una sola chiamata, ad esempio \textit{String1.equals(String2)} in Java. Solidity, nella sua versione 0.8.0.0, non supporta alcuna funzionalità di questo tipo al momento della stesura del presente documento.} \\ \hline
						\textbf{Soluzione} & {La soluzione proposta è basata sull'utilizzo di una funzione di hash per il confronto, opzionalmente combinata con un controllo della corrispondenza della lunghezza delle stringhe fornite, per eliminare fin dall'inizio le coppie con lunghezze differenti.} \\ \hline
						\textbf{Riconoscimento} & {Individuare un uguaglianza fra due risultati della funzione \textit{keccak256()}.} \\ \hline
						\textbf{Snippet di Codice\newline(Versione Semplificata)} & Il codice sorgente di riferimento è consultabile nell'\hyperref[appendix:string_equality_comparison]{Appendice}.  \\ \hline
						\textbf{Riferimenti} & \cite[String Equaliy Comparison]{fravoll} \\
					\end{tabular}
				};
				\draw [rounded corners=.5em] (table.north west) rectangle (table.south east);
			\end{tikzpicture}
			\caption{Specifiche String Equality Comparison Pattern}
		\end{table}
	}
	{\subsection{Tight Variable Packing Pattern}
	Il \textit{Tight Variable Packing} pattern propone un meccanismo per ottimizzare il consumo di gas nella memorizzazione e nella lettura variabili di dimensione statica.\par
	Lo storage in Ethereum è una struttura \textit{chiave-valore} con chiavi e valori di 32 byte ciascuno. Quando viene allocato lo storage di un contratto, tutte le variabili di dimensione statica, eccetto le mappature e gli array di dimensione dinamica, vengono memorizzate nello storage una dopo l'altra, nell'ordine in cui sono state dichiarate.\par I tipi di dati più comunemente utilizzati come \textit{byte32}, \textit{uint} e \textit{int} occupano esattamente uno slot da 32 byte, perciò vengono letti o memorizzati in un’unica operazione.\par
	Sarebbe, quindi, ottimale ordinare la dichiarazione dei tipi di dimensione statica più piccola, come \textit{byte16}, \textit{uint8} e così via, in modo che l’EVM possa raggrupparli in un singolo slot da 32 byte e operare su di essi con un'unica operazione, utilizzando meno memoria e risparmiando gas.
	\begin{table}[H]
		\centering
		\begin{tikzpicture}
			\node (table) [inner sep=0pt] {
				\def\arraystretch{1.5}
				\begin{tabular}{p{0.30\linewidth} | p{0.65\linewidth}}
					\textbf{Problema} & {Salvare o leggere dati di dimensione inferiori a 32 byte può causare uno spreco di gas correlato all’esecuzione di operazioni di lettura e scrittura superflue su uno storage basato su slot di 32 byte.} \\ \hline
					\textbf{Soluzione} & {Ordinare la dichiarazione, dei tipi più piccoli come \textit{bytes16} \textit{uint8}, e così via, in modo che l’EVM possa raggrupparli in un singolo slot da 32 byte.} \\ \hline
					\textbf{Riconoscimento} & {Individuare la definizione di una \textit{struct} contenente tipi di dati la cui somma dello spazio occupato sia minore o uguale, quando possibile, a 32 byte.} \\ \hline
					\textbf{Snippet di Codice\newline(Versione Semplificata)} & Il codice sorgente di riferimento è consultabile nell'\hyperref[appendix:tight_variable_packing]{Appendice}.  \\ \hline
					\textbf{Riferimenti} & \cite[Tight Variable Packing]{fravoll} \cite[Packing Variables]{9050163} \\
				\end{tabular}
			};
			\draw [rounded corners=.5em] (table.north west) rectangle (table.south east);
		\end{tikzpicture}
		\caption{Specifiche Tight Variable Packing Pattern}
	\end{table}
}
}

{\section{Lifecycle Design Pattern}
	I Lifecycle design pattern propongono meccanismi per la creazione e la distruzione degli smart-contract. \par
	Si individuano i seguenti design pattern: \textit{Auto Deprecation} e \textit{Mortal}.
	
	{\subsection{Auto Deprecation Pattern}
		L’\textit{Auto Deprecation} pattern propone un meccanismo che, definito uno specifico quanto di tempo, permette di interrompere automaticamente, allo scadere del quanto di tempo, l'esecuzione di specifiche funzionalità del contratto.
		\begin{table}[H]
			\centering
			\begin{tikzpicture}
				\node (table) [inner sep=0pt] {
					\def\arraystretch{1.5}
					\begin{tabular}{p{0.30\linewidth} | p{0.65\linewidth}}
						\textbf{Problema} & {Si immagini uno scenario d’uso come un evento a tempo, allo scadere di un quanto di tempo, lo smart-contract deve interrompere l’esecuzione della funzionalità relativa all'evento.} \\ \hline
						\textbf{Soluzione} & {Definire un quanto di tempo e un modifier che controlli la scadenza, applicare il modifier nella definizione delle funzioni soggette a scadenza.} \\ \hline
						\textbf{Riconoscimento} & {Data la libertà implementativa, il riconoscimento di questo design pattern si basa sulla ricerca della definizione e uso di una funzione di controllo della scadenza denominata \textit{expired()} e dei modifier relativi.} \\ \hline
						\textbf{Snippet di Codice\newline(Versione Semplificata)} & Il codice sorgente di riferimento è consultabile nell'\hyperref[appendix:auto_deprecation]{Appendice}.  \\ \hline
						\textbf{Riferimenti} & \cite[AutomaticDeprecation]{maxwoe} \cite[AutoDeprecation]{cjgdev} \\
					\end{tabular}
				};
				\draw [rounded corners=.5em] (table.north west) rectangle (table.south east);
			\end{tikzpicture}
			\caption{Specifiche Auto Deprecation Pattern}
		\end{table}
	}
	{\subsection{Mortal Pattern}
		Il \textit{Mortal} pattern propone un meccanismo terminare e distruggere un contratto.\par
		Poiché è la distruzione di un contratto è un operazione critica, la funzionalità è riservata al proprietario del contratto e per questo motivo l’accesso a tale metodo viene spesso gestito attraverso un Authorization pattern design.
		\begin{table}[H]
			\centering
			\begin{tikzpicture}
				\node (table) [inner sep=0pt] {
					\def\arraystretch{1.5}
					\begin{tabular}{p{0.30\linewidth} | p{0.65\linewidth}}
						\textbf{Problema} & {Generalmente, l'esistenza e l'esecuzione di un contratto sono legate all'esistenza della rete Ethereum stessa. Potrebbe, però, in certi casi, come la terminazione del servizio offerto, essere necessario terminare e distruggere il contratto.} \\ \hline
						\textbf{Soluzione} & {Solidity mette a disposizione la funzione \textit{selfdestruct(address)} che termina il contratto che la esegue, non prima di aver trasferito il fondo monetario del contratto all'indirizzo fornito come parametro.} \\ \hline
						\textbf{Riconoscimento} & {Individuare l'esecuzione della funzione \textit{selfdestruct(address)}. Al fine di evidenziare \textit{selfdestruct(address)} come fulcro del pattern design, l'elemento di autenticazione non viene considerato.} \\ \hline
						\textbf{Snippet di Codice\newline(Versione Semplificata)} & Il codice sorgente di riferimento è consultabile nell'\hyperref[appendix:mortal]{Appendice}.  \\ \hline
						\textbf{Riferimenti} & \cite[Mortal]{maxwoe} \cite[Mortal]{cjgdev} \\
					\end{tabular}
				};
				\draw [rounded corners=.5em] (table.north west) rectangle (table.south east);
			\end{tikzpicture}
			\caption{Specifiche Mortal Pattern}
		\end{table}
	}
}



\begin{frontmatter}
\pagenumbering{arabic} \setcounter{page}{1}
\chapter{Conclusioni}
Termina così lo sviluppo dell'applicativo software proposto nella tesi.\\
\newline
\textit{Solidity Design Pattern Analyzer} è un software \textit{open-source} rilasciato con licenza \textit{MIT}, il cui codice sorgente è liberalmente accessibile nel relativo repository su GitHub\cite{github-repo}.\\
\newline
L'applicativo software è in grado di eseguire le seguenti operazioni:
\begin{itemize}
	\item Rilevare, nei limiti linguistici e delle dipendenze utilizzate, tutti e ventidue i design pattern documentati nella tesi, i cui relativi descriptor sono inclusi nel codice sorgente, ed è possibile, mediante la combinazione di controlli generici, definire nuovi descriptor per riconoscere design pattern futuri;
	\item Descrivere uno smart-contract, ovvero estrarre le informazioni utili a creare un nuovo descriptor;
\end{itemize} 
Si conclude la tesi elencando un'ipotetica serie di sviluppi futuri:
\begin{itemize}
	\item Introdurre il supporto di dizionari di lingue diverse dall'inglese al fine di permettere l'analisi degli smart-contract le cui informazioni utili, come nomi di variabili di stato o di funzioni, corrispondono a parole di lingue straniere;
	\item Implementare un sistema di raggruppamento e interdipendenza dei controlli allo scopo di rilevare costrutti più articolati;
	\item Progettare un'interfaccia grafica per un utilizzo più user-friendly dell'applicativo;
\end{itemize}
 
 
\printbibliography[heading=bibintoc]
\appendix
\chapter{Appendice}
\section{Codice di riferimento per Ownership}\label{appendix:ownership}
{\begin{lstlisting}[language=Solidity, caption={Codice di riferimento per Ownership}]
contract Ownable {
	address private _owner;
	event OwnershipTransferred(address indexed previousOwner,address indexed newOwner);
			
			constructor() {
				_owner = msg.sender;
				emit OwnershipTransferred(address(0), _owner); }
			
			modifier onlyOwner() {
				require(_owner == msg.sender, "Ownable: caller is not the owner");
				_; }
			
			function renounceOwnership() public virtual onlyOwner {
				emit OwnershipTransferred(_owner, address(0));
				_owner = address(0); }
			
			function transferOwnership(address newOwner) public virtual onlyOwner {
				require(
				newOwner != address(0),
				"Ownable: new owner is the zero address"
				);
				emit OwnershipTransferred(_owner, newOwner);
				_owner = newOwner; }
}\end{lstlisting}}
\end{frontmatter}
\end{document}
