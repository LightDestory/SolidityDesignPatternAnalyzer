\chapter*{Abstract}\label{abstract}
Blockchain technologies assume a key role in Web3, the next generation of the World Wide Web, whose goal is to create a decentralized and autonomous Internet in which users have greater control and ownership of their online data and activities. Within the Ethereum blockchain, automated digital contracts, called \textit{"smart-contracts"}, are executed when certain conditions are met. Smart-contracts are written in code by a developer and as such are susceptible to vulnerabilities and security problems. A coding error or vulnerability can allow an attacker to compromise the contract and cause extensive and irreparable damage. Therefore, it is important that smart-contracts are carefully written and tested to ensure that they are free of vulnerabilities and function as intended.\\
\\
The goal of this thesis is to document what \textit{design patterns} have been designed for Ethereum blockchain smart-contracts and to develop a software application capable of automatically analyzing a smart-contract in order to detect what \textit{design patterns} have been used.
\vspace{15pt}
\begin{center}
\large$\star\star\star$
\end{center}
\vspace{15pt}
Le tecnologie blockchain assumono un ruolo chiave nel Web3, la prossima generazione del World Wide Web, il cui obiettivo è creare un Internet decentralizzato e autonomo in cui gli utenti hanno maggiore controllo e proprietà dei propri dati e attività online. All'interno della blockchain Ethereum vengono eseguiti, quando determinate condizioni sono soddisfatte, dei contratti digitali automatizzati, denominati \textit{"smart-contract"}. Gli smart-contract sono scritti in codice da un programmatore e come tali sono suscettibili a vulnerabilità e problemi di sicurezza. Un errore di codifica o una vulnerabilità può permettere a un attaccante di compromettere il contratto e causare danni ingenti e irreparabili. Pertanto, è importante che gli smart-contract vengano scritti e testati con cura per garantire che siano privi di vulnerabilità e funzionino come previsto.\\
\\
L'obiettivo di questa tesi è documentare quali \textit{design pattern} siano stati ideati per gli smart-contract della blockchain Ethereum e sviluppare un applicativo software capace di analizzare automaticamente uno smart-contract al fine di rilevare quali \textit{design pattern} siano stati utilizzati.
 
