\chapter{Strumenti Utilizzati}
L'applicativo software sviluppato per automatizzare il processo di analisi statica del codice sorgente di uno smart-contract, scritto in Solidity, legge il codice sorgente, fornito in input, rileva l'utilizzo di specifici design pattern e infine fornisce, in output, le informazioni ottenute all'utente finale.\par
Per lo sviluppo di tale applicativo si è fatto affidamento a numerose tecnologie.

{\section{Python}

\begin{wrapfigure}{r}{0.20\textwidth}
	\centering
	\includegraphics[scale=0.1]{python-logo}
\end{wrapfigure}
Python\cite{python} è un linguaggio di programmazione ad alto livello con una sintassi semplice e intuitiva. È un linguaggio interpretato con tipizzazione dinamica debole, ovvero la tipizzazione delle variabili avviene durante l'esecuzione del codice. Questo rende Python molto flessibile e facile da utilizzare.\par
Lo sviluppo di Python ha avuto inizio negli anni '90 da Guido van Rossum e si è rapidamente affermato come uno dei linguaggi di programmazione più popolari, grazie sopratutto alla sua versatilità in svariate campi, tra cui web development, sviluppo di software scientifico, automazione, machine learning, statistica e molto altro ancora.\par
Uno dei punti di forza di Python è la sua estesa libreria standard, che include molte funzioni già implementate per molte attività comuni, come l'elaborazione di stringhe, la gestione dei file, la connessione a database, il parsing di XML e JSON. Ad affiancare la libreria standard ci sono anche numerosi sviluppatori che hanno creato molti framework e librerie esterne per molte applicazioni specifiche.\par
In sintesi, Python è un linguaggio di programmazione estremamente potente e versatile che rende facile sviluppare software in molti ambiti diversi. La sua sintassi semplice e intuitiva, la vasta libreria standard e la forte comunità di sviluppatori lo rendono una scelta popolare per molti programmatori.

\subsection{Python-Solidity-Parser}
Alla base della tecnica di analisi statica vi è una componente software denominata \textit{"parser"}. Un parser analizza un input, spesso una stringa o un file testuale, e lo trasforma in una struttura dati più significativa, come un albero sintattico.\par
La struttura dati prodotta da un parser può essere facilmente manipolata o elaborata da altre componenti software, le quali potranno effettuare un'analisi semantica dei dati e pendere decisioni basandosi su di essa.\\ \newline
Il processo di parsing consiste di solito di diverse fasi, tra cui:
\begin{itemize}
	\item \textit{Analisi lessicale}: in cui si analizza l'input carattere per carattere e lo divide in token, ovvero unità più piccole e significative;
	\item \textit{Analisi sintattica}: in cui si analizza la sequenza di token generata dalla fase di analisi lessicale e si verifica che seguano la grammatica, spesso generata con ANTLR\cite{ANTLR}, del linguaggio di input. Se viene rilevato un errore sintattico, il parser solitamente lo segnala e interrompe il processo di parsing;
	\item \textit{Costruzione dell'albero sintattico}: in cui si costruisce un albero sintattico, una struttura dati che rappresenta la struttura sintattica dell'input;
\end{itemize}
Per quanto riguarda Solidity e Python, esiste un parser sperimentale open-source sviluppato in Python denominato \textit{"python-solidity-parser"}\cite{python-solidity-parser}. Non offre tutte le funzionalità tipiche di un parser e risulta parecchio acerbo, ma rappresenta una componente fondamentale per lo sviluppo dell'applicativo della tesi.
}
{\section{JSON}
	
	\begin{wrapfigure}{l}{0.20\textwidth}
		\centering
		\includegraphics[scale=0.09]{json-logo}
	\end{wrapfigure}
	JSON (\textit{JavaScript Object Notation}) è un formato di testo semplice e leggero per lo scambio di dati tra sistemi informatici. JSON è stato progettato per essere facile da leggere e scrivere sia per le macchine sia per gli esseri umani, può essere facilmente manipolato ed elaborato da molti linguaggi di programmazione come Python, Java, JavaScript e molti altri. Per questo motivo è diventato uno dei formati di dati più diffusi per l'integrazione tra applicazioni e il trasporto di dati tramite internet. \par
	I dati sono rappresentati come coppie chiave-valore, dove la chiave è una stringa e il valore può essere una stringa, un numero, un booleano, un array o un altro oggetto JSON. Gli oggetti JSON sono delimitati da parentesi graffe "\{\}" e i membri dell'oggetto sono separati da virgole. Gli array JSON, invece, sono delimitati da parentesi quadre "[]" e i membri dell'array sono separati da virgole.
	
	{\begin{lstlisting}[language=json, caption={Esempio di oggetto JSON}]
{
	"nome": "Mario",
	"cognome": "Rossi",
	"eta": 30,
	"genere": "M",
	"indirizzo": {
		"via": "Via Roma",
		"citta": "Roma",
		"paese": "Italia"
	},
	"interessi": ["cinema", "sport", "lettura"]
}\end{lstlisting}}
	
	\subsection{JSON Schema Draft-07}
	JSON Schema\cite{JSON-Schema} è un linguaggio formale per descrivere la struttura di un documento JSON. Permette di definire un modello di dati, un insieme di regole e restrizioni, con cui convalidare una struttura dati JSON e verificare che sia ben formattata, al fine di prevenire errori e garantire la qualità dei dati. Tramite un modello di dati si potrebbe richiedere la presenza obbligatoria di determinati campi, la lunghezza minima o massima di una stringa, o il tipo di dati che un determinato campo può contenere.\par
	Il draft-07\cite{JSON-Schema-Draft-07} è la settima specifica del formato JSON Schema, viene utilizzata questa versione poiché la libreria di validazione per Python ha un supporto parziale per le versioni successive.
{\begin{lstlisting}[language=json, caption={Esempio di modello di dati JSON-Schema}]
{
	"$schema": "http://json-schema.org/draft-07/schema#",
	"type": "object",
	"properties": {
		"nome": {
			"type": "string"
		},
		
		"cognome": {
			"type": "string"
		},
		"eta": {
			"type": "integer"
		},
		"genere": {
			"type": "string"
		},	
	},
	"required": ["nome", "cognome", "eta"]
}\end{lstlisting}}
}